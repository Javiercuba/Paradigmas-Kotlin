% Prof. Dr. Ausberto S. Castro Vera
% UENF - CCT - LCMAT - Curso de Ci\^{e}ncia da Computa\c{c}\~{a}o
% Campos, RJ,  2020
% Disciplina: Paradigmas de Linguagens de Programa\c{c}\~{a}o
% Aluno: Javier Ernesto


\chapterimage{img_superior} % Chapter heading image
\chapter{ Introdu\c{c}\~{a}o}

Kotlin é uma linguagem de programação multiplataforma,
multiparadigma, consistente e de tipagem estática, 
desenvolvido pela empresa JetBrains em 2011, que é
compilado e executado em ambiente Java na JVM 
(Java Virtual Machine). 

Ele se beneficia do aprendizado adquirido como algumas decisões de design tomadas em Java e outras línguas,
como Scala. Ele evoluiu além do que era possível com línguas mais antigas e tem
corrigido o que era doloroso sobre eles.

Atualmente a linguagem é mais 
abrangente, em fevereiro de 2012 a JetBrains  o transformou
em um projeto open source através da licença apache 2.


\section{Aspectos hist\'{o}ricos da linguagem Kotlin}
Em 2011 a linguagem de programaç\~{a}o Kotlin foi anunciada pela JetBrnais como alternativa
de escrever códigos como nas linguagens Java ou Scala que possa ser executada na Máquina Virtual do Java
(JVM). Seis anos depois o Google anunciou que Kotlin seria um caminho de 
desenvolvimento oficialmente suportado para o sistema operacional do Android.\cite{fazio2021kotlin} 

\section{\'{A}reas de Aplica\c{c}\~{a}o da Linguagem}
Ao ser compatível com a JVM (a máquina virtual do Java), 
Kotlin se torna viável como linguagem para aplicações web, 
Android, Desktop, macOS nativo e aplicativos Windows assim como Java.

Esses dois fatores são pontos de ruptura que fizeram os
desenvolvedores Android rapidamente adotar o idioma:
\begin{itemize}
	\item Kotlin é muito intuitivo e facil de aprender para desenvolvedores Java.
	A maior parter das linguagens são muito similares oa que nós ja sabemos, e as diferenças
   serão dominadas ao longo do tempo.
	
	\item Total integração com a IDE. Android Studio consegue entender,
    compilar e rodar Kotlin. Além disso, o apoio para isso
    linguagem vem da empresa que desenvolve o IDE, então nós como desenvolvedores Android
     somos cidadãos de primeira classe. 
	
\end{itemize}

Isso também quer dizer que podemos utilizar Java e Kotlin em um mesmo projeto.

Estima-se que com Kotlin a quantidade de código escrito
cai cerca de 40\% em relação ao Java. 

\subsection{ Programa\c{c}\~{a}o Cient\'{\i}fica}
A Programação Científica é uma área de estudo que,
utilizando computadores, se interessa pela
construção de modelos matemáticos e pelas técnicas 
para determinar soluções numéricas, na
análise e resolução de problemas reais
(científicos e da engenharia)

A Programação consiste, em termos práticos, na aplicação da 
simulação computacional e de outras formas de
computação, na análise e resolução de problemas 
reais em várias áreas científicas e tecnológicas.

Um exemplos de Programação Científica onde temos que implementar um 
algoritmo para calcular a norma-2 (norma Euclidiana) de um vetor x
de tamanho n, definida pela expressão (\ref*{equ1}):

\begin{center}
   \begin{equation}
      ||x||_{2} = \sqrt[]{\sum^{n}_{j=1} |x_{j}|^{2}}
      \label{equ1}
   \end{equation}
\end{center}