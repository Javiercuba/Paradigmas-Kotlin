% Prof. Dr. Ausberto S. Castro Vera
% UENF - CCT - LCMAT - Curso de Ci\^{e}ncia da Computa\c{c}\~{a}o
% Campos, RJ,  2020
% Disciplina: Paradigmas de Linguagens de Programa\c{c}\~{a}o
% Aluno: Javier Ernesto


\chapterimage{img_superior} % Chapter heading image ==>  Trocar este arquivo por outro 1200x468
\chapter{ Programa\c{c}\~{a}o em Kotlin}


%%%%%%%%======================
\section{Expressões condicionais}
O condicional em Kotlin normalmente não é diferente das outras linguagens de programação:
\begin{lstlisting}[label={lst:example1}, language=Kotlin]
  fun maxOf(a: Int, b: Int): Int {
    if (a > b) {
        return a
    } else {
        return b
    }
}
  \end{lstlisting}
É possivel ser usado também como uma expressão:

\begin{lstlisting}[label={lst:example1}, language=Kotlin]
    fun maxOf(a: Int, b: Int) = if (a > b) a else b
  \end{lstlisting}


%%%%%%%%======================

%%%%%%%%======================
\section{Expressões de repetição}
\subsection{For}
Kotlin possui duas formas de utilização do \emph{For}:
\begin{lstlisting}[label={lst:example1}, language=Kotlin]
  val items = listOf("maça", "banana", "abacaxi")
  \end{lstlisting}
\begin{itemize}
  \item Nesse caso o \emph{for} ja percorre a lista toda sem
        necessidade de receber um valor de parada como normalmente é visto em outras linguagens: 
        \begin{lstlisting}[label={lst:example1}, language=Kotlin]
    for (item in items) {
    println(item)  
    } \* Saida: maça 
               banana 
               abacaxi */
    \end{lstlisting}
        
  \item Caso seja necessário usar o indice da lista pode ser feito dessa maneira:
        \begin{lstlisting}[label={lst:example1}, language=Kotlin]
    for (index in items.indices) {
      println("Posição $index é ${items[index]}")
    }//   
    \* Saida: Posição 0 é maça 
              Posição 1 é banana 
              Posição 2 é abacaxi */  
  
  \end{lstlisting}
\end{itemize}

\subsection{While}
O While em Kotlin funciona igual as outras linguagens, uma estrutura de parada e um contador no final.  
\begin{lstlisting}[label={lst:example1}, language=Kotlin]
  while (index < items.size) {
    println("item at $index is ${items[index]}")
    index++
}
\end{lstlisting}


\subsection{Range}
Range é usado para verificar se um valor está presente dentro de um conjunto ou valores:

\begin{lstlisting}[label={lst:example1}, language=Kotlin]
  val x = 5
  val y = 9
  if (x in 1..y) {
      println("dentro")
  }\\ dentro
\end{lstlisting}
Pode ser usado como ums forma de loop também:
\begin{lstlisting}[label={lst:example1}, language=Kotlin]
  for (x in 1..5) {
    print(x)
}\\ 12345
\end{lstlisting}
Ou como uma Progressão:
\begin{lstlisting}[label={lst:example1}, language=Kotlin]
  for (x in 1..10 step 2) {
    print(x)
}\\13579
\end{lstlisting}

%%%%%%%%======================

%%%%%%%%======================
\section{Funções}
As funções em Kotlin começam sempre com a palavra \emph{fun}. 
\begin{lstlisting}[label={lst:example1}, language=Kotlin]
  fun sum(a: Int, b: Int): Int {
    return a + b
}
  \end{lstlisting}
%%%%%%%%======================
\subsection{Parâmetros}
Os parâmetros da função são definidos usando a
notação Pascal -\emph{nome : tipo }. Os parâmetros são separados por vírgulas e cada parâmetro deve ser digitado explicitamente:
\begin{lstlisting}[label={lst:example1}, language=Kotlin]
  fun sum(a: Int, b: Int): Int {/*...*/}
  \end{lstlisting}
%%%%%%%%======================

%%%%%%%%======================
\subsection{Argumentos padrão}
Os parâmetros da função podem ter valores padrão, que são usados ​​quando você ignora o argumento correspondente. Isso reduz o número de sobrecargas:
\begin{lstlisting}[label={lst:example1}, language=Kotlin]
  fun read(
    b: ByteArray,
    off: Int = 0,
    len: Int = b.size,
) { /*...*/ }
  \end{lstlisting}
Um valor padrão é definido usando \emph{=} após o tipo.

Os métodos de substituição sempre usam os mesmos valores de parâmetro padrão que o método base. Ao substituir um método que tem valores de parâmetro padrão, os valores de parâmetro padrão devem ser omitidos da assinatura:
\begin{lstlisting}[label={lst:example1}, language=Kotlin]
    open class A {
      open fun foo(i: Int = 10) { /*...*/ }
    }

    class B : A() {
      override fun foo(i: Int) { /*...*/ }  
      // Sem valor padrao alocado.
    }
    \end{lstlisting}

Se um parâmetro padrão precede um parâmetro sem valor padrão, o valor padrão só pode ser usado chamando a função com argumentos nomeados :

\begin{lstlisting}[label={lst:example1}, language=Kotlin]
      fun foo(
    bar: Int = 0,
    baz: Int,
    ) { /*...*/ }

    foo(baz = 1) // O valor padrao para é bar = 0.
      \end{lstlisting}

Se o último argumento após os parâmetros padrão for um lambda , você pode passá-lo como um argumento nomeado ou fora dos parênteses :
\begin{lstlisting}[label={lst:example1}, language=Kotlin]
        fun foo(
          bar: Int = 0,
          baz: Int = 1,
          qux: () -> Unit,
      ) { /*...*/ }
      
      foo(1) { println("hello") }     
      // Usando o valor padrao de baz = 1
      
      foo(qux = { println("hello") }) 
      // Usando ambos os valores padroes bar = 0 e baz = 1
      
      foo { println("hello") }        
      // Usando ambos os valores padroes bar = 0 e baz = 1
        \end{lstlisting}

%%%%%%%%======================

%%%%%%%%======================
\subsection{Argumentos nomeados}

Ao chamar uma função, você pode nomear um ou mais de seus argumentos. Isso pode ser útil quando uma função tem muitos argumentos e é difícil associar um valor a um argumento, especialmente se for um booleano ou nullvalor.

Ao usar argumentos nomeados em uma chamada de função, você pode alterar livremente a ordem em que são listados e, se quiser usar seus valores padrão, pode simplesmente deixar esses argumentos de fora.  

Considerando a seguinte função, \emph{reformat()} que possui 4 argumentos com valores padrão.
\begin{lstlisting}[label={lst:example1}, language=Kotlin]
  fun reformat(
    str: String,
    normalizeCase: Boolean = true,
    upperCaseFirstLetter: Boolean = true,
    divideByCamelHumps: Boolean = false,
    wordSeparator: Char = ' ',
) { /*...*/ }
  \end{lstlisting}

Ao chamar esta função, você não precisa nomear todos os seus argumentos:

\begin{lstlisting}[label={lst:example1}, language=Kotlin]
    reformat(
    "String!",
    false,
    upperCaseFirstLetter = false,
    divideByCamelHumps = true,
    '_'
)
    \end{lstlisting}

Você pode pular todos aqueles com valores padrão:
\begin{lstlisting}[label={lst:example1}, language=Kotlin]
      reformat("This is a long String!")
      \end{lstlisting}

Você também pode pular argumentos específicos com valores padrão, em vez de omitir todos eles. No entanto, após o primeiro argumento ignorado, você deve nomear todos os argumentos subsequentes:
\begin{lstlisting}[label={lst:example1}, language=Kotlin]
     reformat("This is a short String!",
     upperCaseFirstLetter = false,
     wordSeparator = '_')
        \end{lstlisting}
Você pode passar um número variável de argumentos ( vararg) com nomes usando o spreadoperador:
\begin{lstlisting}[label={lst:example1}, language=Kotlin]
     fun foo(vararg strings: String) { /*...*/ }
     foo(strings = *arrayOf("a", "b", "c"))
          \end{lstlisting}
%%%%%%%%======================


%%%%%%%%======================
\subsection{Funções de retorno de unidade}
Se uma função não retorna um valor útil, seu tipo de retorno é \emph{Unit}. \emph{Unit} é um tipo com apenas um valor - \emph{Unit}. Este valor não precisa ser retornado explicitamente:
\begin{lstlisting}[label={lst:example1}, language=Kotlin]
  fun printHello(name: String?): Unit {
    if (name != null)
        println("Hello $name")
    else
        println("Hi there!")
    // `return Unit` ou `return` é opcional
}
       \end{lstlisting}
A \emph{Unit} declaração do tipo de retorno também é opcional. O código acima é equivalente a:
\begin{lstlisting}[label={lst:example1}, language=Kotlin]
        fun printHello(name: String?) { ... }
             \end{lstlisting}
%%%%%%%%======================

\subsection{Funções de expressão única}
Quando uma função retorna uma única expressão, as chaves podem ser omitidas e o corpo é especificado após um \emph{=} símbolo:
\begin{lstlisting}[label={lst:example1}, language=Kotlin]
 fun double(x: Int): Int = x * 2
       \end{lstlisting}
Declarar explicitamente o tipo de retorno é opcional quando pode ser inferido pelo compilador:      
\begin{lstlisting}[label={lst:example1}, language=Kotlin]
 fun double(x: Int) = x * 2
       \end{lstlisting}

\subsection{Funções locais}
Kotlin oferece suporte a funções locais, que são funções dentro de outras funções:
\begin{lstlisting}[label={lst:example1}, language=Kotlin]
  fun dfs(graph: Graph) {
    fun dfs(current: Vertex, visited: MutableSet<Vertex>) {
        if (!visited.add(current)) return
        for (v in current.neighbors)
            dfs(v, visited)
    }

    dfs(graph.vertices[0], HashSet())
}
        \end{lstlisting}
Uma função local pode acessar variáveis ​​locais de funções externas (o fechamento). No caso acima, \emph{visited} pode ser uma variável local:
\begin{lstlisting}[label={lst:example1}, language=Kotlin]
   fun dfs(graph: Graph) {
     val visited = HashSet<Vertex>()
       fun dfs(current: Vertex) {
          if (!visited.add(current)) return
          for (v in current.neighbors)
               dfs(v)
        }
        
            dfs(graph.vertices[0])
    }
                \end{lstlisting}

%%%%%%%%======================

\section{Classes}
Classes em Kotlin são declaradas usando-se uma palavra-chave
A declaração da classe consiste no nome da classe, no cabeçalho da
classe (especificando seus parâmetros de tipo, o construtor
primário e algumas outras coisas) e o corpo da classe entre
chaves. O cabeçalho e o corpo são opcionais; se a classe não tiver corpo, as chaves podem ser omitidas.
\begin{lstlisting}[label={lst:example1}, language=Kotlin]
      class Pessoa { /*...*/ }
      \end{lstlisting}
%%%%%%%%======================


%%%%%%%%======================
\section{Construtores}
Uma classe em Kotlin pode ter um construtor primário ou mais
construtores secundários. O construtor primário é uma parte do
cabeçalho da classe e vai depois do nome da classe e dos parâmetros
de tipo opcionais.
\begin{lstlisting}[label={lst:example1}, language=Kotlin]
  class Pessoa constructor(firstName: String) { /*...*/ }
      \end{lstlisting}

Se o construtor primário não tiver anotações ou modificadores 
de visibilidade, a constructorpalavra - chave pode ser omitida:
\begin{lstlisting}[label={lst:example1}, language=Kotlin]
  class Pessoa(firstName: String) { /*...*/ }
      \end{lstlisting}
%%%%%%%%======================


%%%%%%%%======================
\section{Construtores Secundarios}
Uma classe em Kotlin pode ter um construtor primário ou mais
construtores secundários. O construtor primário é uma parte do
cabeçalho da classe e vai depois do nome da classe e dos parâmetros
de tipo opcionais.
\begin{lstlisting}[label={lst:example1}, language=Kotlin]
  class Pessoa constructor(val pets: MutableList<Pet> = mutableListOf())) {
    class Pet {
      constructor(owner: Pessoa) {
          owner.pets.add(this) 
      }
  }
      \end{lstlisting}

Se a classe tem um construtor primário, cada 
construtor secundário precisa delegar ao construtor
primário, direta ou indiretamente por meio de outro (s) construtor (es) secundário (s). A delegação a outro construtor da mesma classe é feita usando a thispalavra-chave:
\begin{lstlisting}[label={lst:example1}, language=Kotlin]
        class Pessoa(val name: String) {
          val children: MutableList<Pessoa> = mutableListOf()
          constructor(name: String, parent: Pessoa) : this(name) {
              parent.children.add(this)
          }
      }
      \end{lstlisting}
O código nos blocos inicializadores torna-se efetivamente parte do construtor primário. 
A delegação ao construtor primário acontece como a primeira instrução de um construtor 
secundário, portanto, o código em todos os blocos inicializadores e inicializadores de propriedade
é executado antes do corpo do construtor secundário.

Mesmo que a classe não tenha um construtor primário, a delegação ainda acontece implicitamente e os blocos inicializadores ainda são executados:
\begin{lstlisting}[label={lst:example1}, language=Kotlin]
        class Constructors {
          init {
              println("bloco inicial")
          }
          constructor(i: Int) {
              println("Constructor $i")
          }
        }
      fun main{
       Constructors(1)
      } \\Construtor 1 do bloco de inicialização
      \end{lstlisting}

Se uma classe não abstrata não declara nenhum construtor (primário ou secundário), ela terá um construtor primário gerado sem argumentos. A visibilidade do construtor será pública.

Se você não quiser que sua classe tenha um construtor público, declare um construtor primário vazio com visibilidade não padrão:








%%%%%%%%======================

%%%%%%%%======================


