% Prof. Dr. Ausberto S. Castro Vera
% UENF - CCT - LCMAT - Curso de Ci\^{e}ncia da Computa\c{c}\~{a}o
% Campos, RJ,  2020
% Disciplina: Paradigmas de Linguagens de Programa\c{c}\~{a}o
% Aluno: Javier Ernesto


\chapterimage{img_superior} % Chapter heading image ==>  Trocar este arquivo por outro 1200x468
\chapter{ Programa\c{c}\~{a}o em Kotlin}


%%%%%%%%======================
\section{Expressões condicionais}

\begin{lstlisting}[label={lst:example1}, language=Kotlin]
  fun maxOf(a: Int, b: Int): Int {
    if (a > b) {
        return a
    } else {
        return b
    }
}
  \end{lstlisting}
  Em Kotlin o condicional \emph{if} pode ser usado também como uma expressão:

  \begin{lstlisting}[label={lst:example1}, language=Kotlin]
    fun maxOf(a: Int, b: Int) = if (a > b) a else b
  \end{lstlisting}
  %%%%%%%%======================

%%%%%%%%======================

%%%%%%%%======================
\section{Funções}
As funções em Kotlin começam sempre com a palavra \emph{fun}. Os
parametros da função vem a variavel primeiro e em seguida vem o tipo. 
\begin{lstlisting}[label={lst:example1}, language=Kotlin]
  fun sum(a: Int, b: Int): Int {
    return a + b
}
  \end{lstlisting}

%%%%%%%%======================

\section{Classes}
Classes em Kotlin são declaradas usando-se uma palavra-chave
A declaração da classe consiste no nome da classe, no cabeçalho da
classe (especificando seus parâmetros de tipo, o construtor
primário e algumas outras coisas) e o corpo da classe entre
chaves. O cabeçalho e o corpo são opcionais; se a classe não tiver corpo, as chaves podem ser omitidas.
\begin{lstlisting}[label={lst:example1}, language=Kotlin]
      class Pessoa { /*...*/ }
      \end{lstlisting}
%%%%%%%%======================


%%%%%%%%======================
\section{Construtores}
Uma classe em Kotlin pode ter um construtor primário ou mais
 construtores secundários. O construtor primário é uma parte do
  cabeçalho da classe e vai depois do nome da classe e dos parâmetros
   de tipo opcionais.
\begin{lstlisting}[label={lst:example1}, language=Kotlin]
  class Pessoa constructor(firstName: String) { /*...*/ }
      \end{lstlisting}

Se o construtor primário não tiver anotações ou modificadores 
de visibilidade, a constructorpalavra - chave pode ser omitida:
\begin{lstlisting}[label={lst:example1}, language=Kotlin]
  class Pessoa(firstName: String) { /*...*/ }
      \end{lstlisting}
%%%%%%%%======================


%%%%%%%%======================
\section{Construtores Secundarios}
Uma classe em Kotlin pode ter um construtor primário ou mais
 construtores secundários. O construtor primário é uma parte do
  cabeçalho da classe e vai depois do nome da classe e dos parâmetros
   de tipo opcionais.
\begin{lstlisting}[label={lst:example1}, language=Kotlin]
  class Pessoa constructor(firstName: String) { /*...*/ }
      \end{lstlisting}

Se o construtor primário não tiver anotações ou modificadores 
de visibilidade, a constructorpalavra - chave pode ser omitida:
\begin{lstlisting}[label={lst:example1}, language=Kotlin]
  class Pessoa(firstName: String) { /*...*/ }
      \end{lstlisting}
%%%%%%%%======================

\subsection{Entrada e Sa\'{\i}da formatada}


%%%%%%%%======================
\section{Sele\c{c}\~{a}o}
%%%%%%%%======================
Tipos de IF

Select

%%%%%%%%======================
\section{Repeti\c{c}\~{a}o}
%%%%%%%%======================

%%%%%%%%======================
\section{Fun\c{c}\~{o}es}
%%%%%%%%======================



%%%%%%%%======================
\section{M\'{o}dulos e Subprogramas}
%%%%%%%%======================


\begin{lstlisting}
class Rational(n: Int, d: Int) {

    require(d != 0)

    val numer: Int = n
    val denom: Int = d

    def this(n: Int) = this(n, 1) // auxiliary constructor

    override def toString = numer +"/"+ denom

    def add(that: Rational): Rational =
      new Rational(
        numer * that.denom + that.numer * denom,
        denom * that.denom
      )
  }
    \end{lstlisting}


