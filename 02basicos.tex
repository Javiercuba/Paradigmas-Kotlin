% Prof. Dr. Ausberto S. Castro Vera
% UENF - CCT - LCMAT - Curso de Ci\^{e}ncia da Computa\c{c}\~{a}o
% Campos, RJ,  2020
% Disciplina: Paradigmas de Linguagens de Programa\c{c}\~{a}o
% Aluno: Javier Ernesto


\chapterimage{img_superior} % Chapter heading image ==>  Trocar este arquivo por outro 1200x468
\chapter{ Conceitos b\'{a}sicos da Linguagem Kotlin}

Em Kotlin, tudo é um objeto no sentido de que 
podemos chamar funções e propriedades de membros
 em qualquer variável. Alguns tipos podem ter uma 
 representação interna especial por exemplo, 
 números, caracteres e booleanos podem ser 
 representados como valores primitivos no tempo 
 de execução, mas para o usuário eles se parecem
 com classes comuns.
    %%%%%%%%=================================

    \section{Vari\'{a}veis e constantes}
      As variáveis em Kotlin podem ser facilmente 
    definidas como mutáveis (var) ou imutáveis (val).
    A ideia é muito semelhante ao uso \emph{final} em variáveis Java.
     
    Um objeto imutável é um objeto cujo estado não pode mudar 
    após a instanciação. Se você precisar de uma versão 
    modificada do objeto, um novo objeto precisa ser criado.
    Isso torna a programação muito mais robusta e previsível. 
    Em Java, a maioria dos objetos são mutáveis, o que significa
    que qualquer parte do código que tenha acesso ao objeto pode 
    modificá-lo, afetando o resto do aplicativo.

    Assim, a maneira como pensamos sobre codificação muda um pouco em Kotlin 
    se quisermos fazer uso da imutabilidade. Haverá situações (especialmente no Android,
    onde não temos acesso ao construtor de muitas classes) onde não será 
    possível, mas será na maioria das vezes

    Outro ponto é que geralmente não precisamos 
    especificar tipos de objetos, eles serão inferidos a partir do 
    valor, o que torna o código mais limpo e mais rápido para ser modificado.
    \begin{lstlisting}[label={lst:example1}, language=Kotlin]
      val numero = 1 // Inteiro
      val nome = "Exemplo" // String  
    \end{lstlisting}
    Caso seja necessario declarar o tipo da variavel desejada poderá ser feito dessa forma:
    \begin{lstlisting}[label={lst:example1}, language=Kotlin]
      val numero: Int = 1 // Inteiro
      val nome: String = "Exemplo" // String  
    \end{lstlisting}


\section{Tipos de Dados B\'{a}sicos}
  
  \subsection{Inteiros}
  Kotlin fornece um conjunto de tipos embutidos que representam números.
  Para números inteiros, existem quatro tipos com tamanhos diferentes e faixas de valores diferentes.
  
  \begin{table}[ht]
      \centering
      
      \begin{tabular}{p{2cm} p{2.5cm} p{6.0cm} p{6.0cm}|cp{1cm}}
      \textbf{Tipo} \centering & \textbf{Tamanho(Bits)} & \textbf{Valor mínimo} & \textbf{Valor máximo} \\ 
          Byte \centering    & 8  \centering  & -128   & 127                 \\ 
          Short \centering  & 16  \centering & -32768   & 32767             \\ 
          Int \centering    & 32  \centering &  -2,147,483,648 ($-2 ^{31}$)      &2,147,483,648 ($2^{31}-1$)            \\ 
          Long \centering    & 32  \centering &  -9,223,372,036,854,775,808 ($-2 ^{63}$)      &9,223,372,036,854,775,808 ($2^{63}$)\\ 
      
      \end{tabular}
      \label{Tabela1}
      \end{table}
      Todas as variáveis inicializadas com valores inteiros não excedem o valor 
      máximo de ter o tipo inferido . Se o valor inicial exceder esse valor, 
      então o tipo será . Para especificar o valor explicitamente, apendice o sufixo ao valor.
     
            %%%........................

    C\'{o}digo fonte para a linguagem Kotlin:
    \begin{lstlisting}[label={lst:example1}, language=Kotlin]
      val one = 1 // Int
      val threeBillion = 3000000000 // Long
      val oneLong = 1L // Long
      val oneByte: Byte = 1
    \end{lstlisting}

    

    \subsection{Float}
    Para números reais, Kotlin fornece tipos de pontos flutuantes Float e Double
    
    \begin{tabular}{p{1cm} p{2.5cm} p{3.0cm} p{3.0cm} p{3.5cm} cp{2cm}}
      
      \textbf{Tipo} \centering & \textbf{Tamanho(Bits)} & \textbf{Bits significativos} & \textbf{Bits expoentes} & \textbf{Digitos decimais} \\ 
          Float \centering     & 32  \centering         &  24    & 8              & 6-7 \\
          Double \centering    & 64  \centering         &  53  & 11       &15-16  \\
      \end{tabular}
      Por padrão, se você declarar uma variável 
      atribuindo um número do tipo float assim:
      \begin{lstlisting}[label={lst:example1}, language=Kotlin]
      var x = 134.5 
      \end{lstlisting}
      Você criará um objeto e uma variável do tipo Double.
      Se você adicionar um "F" ou "f" ao final do número, 
      um Float será criado em vez disso
      \begin{lstlisting}[label={lst:example1}, language=Kotlin]
      var x = 134.5F 
      \end{lstlisting}
   
      \subsection{Booleanos}
      Variáveis booleanas são usadas para valores que podem 
      ser verdadeiros ou falsos. 

      As operações incorporadas em booleans incluem:
      \begin{itemize}
      \item
      \begin{lstlisting}[label={lst:example1}, language=Kotlin]
Disjunção - ||
      \end{lstlisting}

      \item
      \begin{lstlisting}[label={lst:example1}, language=Kotlin]
Conjunção - &&
      \end{lstlisting}

      \item
      \begin{lstlisting}[label={lst:example1}, language=Kotlin]
Conjunção - !
      \end{lstlisting}
      \end{itemize}
      
      Cria-se um objeto e a 
      variável booleano se declarar uma variável usando o 
      código em seguida:
      \begin{lstlisting}[label={lst:example1}, language=Kotlin]
      val Um: Boolean = true
      val Dois: Boolean = false
      val Null: Boolean? = null
      
      println(Um || Dois)//true
      println(Um && Dois)//false
      println(!Um)//false
      \end{lstlisting}

      \subsection{Caracteres e Strings}
      Variáveis char são usadas para caracteres únicos. 
      Você cria uma variável Char, designando um caractere 
      em citações simples como esta:
      \begin{lstlisting}[label={lst:example1}, language=Kotlin]
        var letra = 'A'
      \end{lstlisting}
      As variáveis de string são usadas para manter
      vários caracteres amarrados. Você cria uma 
      variável String atribuindo os caracteres 
      com aspas duplas:
      \begin{lstlisting}[label={lst:example1}, language=Kotlin]
        var letra = "Livro"
      \end{lstlisting}

      \subsection{Array}
      Arrays são uma coleção de dados homogêneos. 
      Kotlin oferece suporte a matrizes de diferentes tipos 
      de dados. No exemplo a seguir, faremos uso de 
      diferentes arrays.
      \begin{lstlisting}[label={lst:example1}, language=Kotlin]
        val numbers: IntArray = intArrayOf(1, 2, 3, 4, 5)
        println("Ola!! Sou um exemplo de array " + numbers[2])
        //Ola!! Sou um exemplo de array 3
      \end{lstlisting}
      
      \subsection{Coleções}
      Kotlin tem dois tipos de coleção:
      
      \begin{itemize}
        \item Immutable collection -
        Que significa listas, mapas e conjuntos 
      que não podem ser editáveis
        \item Mutable collection -
        Este tipo de coleção é editável
      \end{itemize}
      
      É muito importante ter em mente o tipo de coleção usada em sua aplicação, pois o sistema Kotlin não representa nenhuma diferença específica entre elas.
      \begin{lstlisting}[label={lst:example1}, language=Kotlin] 
      //mutable List 
      val numbers: MutableList<Int> = mutableListOf(1, 2, 3)
     \end{lstlisting}
      \begin{lstlisting}[label={lst:example1}, language=Kotlin]
      // immutable list
      val readOnlyView: List<Int> = numbers 
        \end{lstlisting}

      \section{Operadores e Express\~{o}es em Kotlin}
    Kotlin suporta o conjunto padrão de operações 
    aritméticas sobre números. Eles são declarados 
    como membros de classes apropriadas.
    \begin{lstlisting}[label={lst:example1}, language=Kotlin]
    fun main() {
      println(1 + 2)                //3
      println(2_500_000_000L - 1L) //2499999999
      println(3.14 * 2.71)        //8.5094
      println(10.0 / 3)          //8.5094
    }
    \end{lstlisting}

    \subsection{Divisão de Inteiros}
    A divisão entre números inteiros sempre 
    retorna um número inteiro. Qualquer 
    parte fracionada é descartada.
    \begin{lstlisting}[label={lst:example1}, language=Kotlin]
    val x = 5 / 2
    println(x == 2)//true
    \end{lstlisting}

    \begin{lstlisting}[label={lst:example1}, language=Kotlin]
    val x = 5 / 2
    println(x == 2.5)  
    // ERROR: Operator '==' cannot be applied to 'Double'
    \end{lstlisting}
    No caso acima o algoritmo deu erro pois a variavel \emph{x} foi inicializada
    como inteiro e recebeu um \emph{Double} em seguida.
    
    A divisão entre qualquer tipo de inteiro também será sempre
    verdadeiro, no caso a seguir teremos uma variavel do tipo \emph{Long} 
    e uma do tipo \emph{Inteiro}.
    \begin{lstlisting}[label={lst:example1}, language=Kotlin]
    val x = 5L / 2
    println(x == 2L)//true
    \end{lstlisting}

    Para retornar um tipo Double é possivel converter um dos argumentos para
    Double ou para qualquer tipo de dados usando a conversão \emph{.to(tipo desejado)}, Exemplo: toDouble.
    \begin{lstlisting}[label={lst:example1}, language=Kotlin]
    val x = 5 / 2.toDouble()
    println(x == 2.5)//true
    \end{lstlisting}

\section{Operador Bit-a-Bit}

Operadores bitwise(Bit-a-Bit) são utilizados quando precisamos realizar operações 
a nível de bits com números inteiros, ou seja, trabalhar com sua 
representação binária.

Kotlin fornece um conjunto de operações bitwise em números inteiros.
Eles operam no nível binário diretamente com pedaços da representação 
dos números. As operações bitwise são representadas por funções que 
podem ser chamadas na forma infixa, Exemplo:
\begin{lstlisting}[label={lst:example1}, language=Kotlin]
  val x = (1 shl 2) and 0x000FF000
\end{lstlisting}

Lista dos operadores em kotlin:

\begin{itemize}
	\item shl(bits) - Deslocamento de bits para a esquerda;
	\item shr(bits) - Deslocamento de bits para a esquerda;
	\item and(bits) - Bitwise AND;
	\item or(bits)  - Bitwise OR;
	\item xor(bits) - Bitwise XOR;
	\item inv()     - Inversão Bitwise;
\end{itemize}

\section{Comparação de números}

As operações sobre números de pontos flutuantes são:


\begin{itemize}
\item
  Verificação da igualdade:  
  \begin{lstlisting}[label={lst:example1}, language=Kotlin]
    a == b //'a' igual a 'b'
    a != b //'b' diferente de 'b'
  \end{lstlisting}

  \item
  Operadores de comparação:  
  \begin{lstlisting}[label={lst:example1}, language=Kotlin]
    a < b  // 'a' menor que 'b' 
    a > b  // 'a' maior que 'b'
    a <= b // 'a' menor ou igual que 'b'
    a >= b // 'a' menor ou igual que 'b'
  \end{lstlisting}
\end{itemize}

\section{Tipos Nulos}
Em Kotlin quando inicializada uma variavel
    temos a opção de adicionar um ponto de interrogação
    ao lado do tipo da variavel representando que essa variavel pode
    receber valor \emph{Null}.
    
    \begin{lstlisting}[label={lst:example1}, language=Kotlin]
        var b: String? = "abc" 
        println(b)//abc
        b = null
        print(b)//null
    \end{lstlisting}

